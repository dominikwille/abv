\documentclass[a4paper, parskip=half]{scrartcl}
\usepackage[utf8]{inputenc}
\usepackage[T1]{fontenc}
\usepackage[english, ngerman]{babel}%trennregeln für En und De
\usepackage{libertine}
\usepackage{xstring}
\usepackage{amssymb,amstext,amsmath}
\usepackage{graphicx}
\usepackage{sectsty}
\usepackage{multirow}
\usepackage{dsfont}
\usepackage{amsfonts}
\usepackage{graphics}
\usepackage{float}
\usepackage{dsfont}
\usepackage{hyperref}
\usepackage{caption}

\newcommand\myTitle{\LaTeX - Grundlagen}
\newcommand\myAuthor{Dominik Wille}
\newcommand{\myImage}[1]{
  \begin{center}
    \begin{minipage}{\linewidth}
      \centering
      \makebox[0cm]{\includegraphics[width=\textwidth]{#1}}
    \end{minipage}
    \captionof{figure}{plotted measure values}
  \end{center}
}

\begin{document}
\begin{titlepage}
  \begin{center}
    \vspace{5cm}
    \huge\bfseries
    \myTitle
    \vspace{1cm}

    \large\normalfont von

    \bigskip
    \textbf{\myAuthor}

    \vspace{2cm}
    \myImage{images/typesetting}
  \end{center}
  \vfill
  \enlargethispage{2cm}
  \parbox[t]{0.45\textwidth}{%
    Freie Universität Berlin\\
    Zentraleinrichtung für Datenverarbeitung\\
    Betriebsysteme und Programmieren
  }
  \parbox[t]{0.45\textwidth}{\raggedleft%
    Dozent: \\
    Dr.\,Herbert Voß\\
  }
\end{titlepage}
%% \maketitle
\section{Vorwort}
Bei diesem Dokument handelt es sich um eine im Ramen des ABV-Kurses Betriebsysteme und Programmieren der
Freien-Universität-Berlin angefärtigte Hausarbeit, die die erlernten Grundlagen von Latex zusammenfasst.

Im folegenden werde ich sowohl auf den Textsatz mit Latex als auch auf einen meiner meinung nach sinnvollen
Arbeitsweg mit Latex eingehen.

\section{Textsatz}

\section{Bilder}
\section{Mathematiksatz}
\section{Tabellen}
\section{Eigene Kommandos}
\section{Arbeiten im Team}
\end{document}
