\documentclass[a4paper, parskip=half]{scrartcl}
\usepackage[utf8]{inputenc}
\usepackage[T1]{fontenc}
\usepackage[english, ngerman]{babel}%trennregeln für En und De
\usepackage{libertine}
\usepackage{xstring}
\usepackage{amssymb,amstext,amsmath}
\usepackage{graphicx}
\usepackage{sectsty}
\usepackage{multirow}
\usepackage{dsfont}
\usepackage{amsfonts}
\usepackage{graphics}
\usepackage{float}
\usepackage{dsfont}
\usepackage{hyperref}
\usepackage{caption}
\usepackage{ifthen}

\newcommand{\myImage}[2]{
  \begin{center}
    \begin{minipage}{\linewidth}
      \centering
      \makebox[0cm]{\includegraphics[width=\textwidth]{#1}}
    \end{minipage}
    \ifthenelse{ \equal{#2}{}} {
      
    } {
    \captionof{figure}{#2}
    }
  \end{center}
}

\newcommand\myTitle{\LaTeX - Grundlagen}
\newcommand\myAuthor{Dominik Wille}
\newcommand\myDate{22 Oktober 2013}

\begin{document}
\begin{titlepage}
  \begin{center}
    \vspace{5cm}
    \huge\bfseries
    \myTitle
    \vspace{1cm}

    \large\normalfont von

    \bigskip
    \textbf{\myAuthor}

    \myDate

    \vspace{2cm}
    \myImage{images/typesetting}{}
  \end{center}
  \vfill
  \enlargethispage{2cm}
  \parbox[t]{0.45\textwidth}{%
    Freie Universität Berlin\\
    Zentraleinrichtung für Datenverarbeitung\\
    Betriebsysteme und Programmieren
  }
  \parbox[t]{0.45\textwidth}{\raggedleft%
    Dozent: \\
    Dr.\,Herbert Voß\\
  }
\end{titlepage}
%% \maketitle
\section{Vorwort}
Bei diesem Dokument handelt es sich um eine im Ramen des ABV-Kurses Betriebsysteme und Programmieren der
Freien-Universität-Berlin angefärtigte Hausarbeit, die die erlernten Grundlagen von Latex zusammenfasst.

Im folegenden werde ich sowohl auf den Textsatz mit Latex als auch auf einen meiner meinung nach sinnvollen
Arbeitsweg mit Latex eingehen.

Desweiteren möchte ich dieses Dokument möglichst so gestalten, dass viele wiederkehrende Elemete in Latex in
Makros gekaplselt werden und somit für weitere Dokumente wiederverwendet werden können.

\section{Textsatz}
Historisch bedigt ist Tex bzw. Latex vorallem für einen optimierten Textsatz konzipiert. Sehr oft ist es hilfreich
eine automatische Worttreenung zu aktivieren um einen schönen Blocksatz zu erreichen. Das \verb+babel+ Paket
liefert für viele Sprachen eine automatische Silbenerkennung, die eine korrekte Worttrennung ermöglicht. Um
das Paket einzubinden genügt ein simples
\begin{verbatim}
\usepackage[english, ngerman, ...]{babel} 
\end{verbatim}
in der Präambel. Einer der großen Vorteile bei der Arbeit mit Latex ist, dass man ein semantisch korrektes
Dokument schreibt und die Darstellung vom Dokument ersteinmal nichts mit deren Inhalt zutun hat. Um Teile
des Textes \textit{kursiv}, \textbf{fett} oder \underline{unterstrichen} darzustellen bietet sich die Verwendung 
von den einfachen Kommandos 
\begin{verbatim}
\textbf{fett gesetzter Text}
\textit{kursiv gesetzter Text}
\underline{unterstricherner Text}
\end{verbatim}
an. 
\newpage
\section{Bilder}
Bilder können in Latex mit dem \verb+graphicx+ Paket eingebunden werden, dieses unterstützt gängige
Bildformate wie \textit{png, jpeg} aber auch Graphiken im \textit{pdf} Format können eingebunden werden.
das zugehöhrige Kommando lautet:
\begin{verbatim}
\includegraphics{/pfad/zur/bilddatei}
\end{verbatim}
Zu beachten ist dabei, dass die Dateiendung nicht anggeben werden muss. Um also beispielsweise \textit{bsp.jpeg}
einzubinden reicht \verb+\includegraphics{bsp}+.
\section{Mathematiksatz}
Da ich im Physikstudium diverse Protokolle mit viel mthematischem Inhalt schreibe möchte ich auch die
Grundlagen vom Mathematiksatz in Latex erkären.

Auch ohne zusätzliches Paket bietet Latex schon einen Mathematikmodus, der über den \$-Delimiter eingeleitet und beendet wird. als Beispiel liefert
\begin{verbatim}
$a^2 + b^2 = c^2$
\end{verbatim}
folgende Ausgabe:
$a^2 + b^2 = c^2$. Um auch ganze Rechnung oder Formelsammlungen in Latex einzubetten gibt es das \verb+amsmath+
Paket. Dieses stellt eine Umgebung für Gleichungen und für Formelblöcke zur Verfügung. benutzt werden diese wie
folgt:
\begin{verbatim}

\end{verbatim}
\section{Tabellen}
\section{Eigene Kommandos}
\section{Verweise}
\section{Arbeiten im Team}
\end{document}
