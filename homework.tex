\documentclass[a4paper, parskip=half]{scrartcl}

\usepackage[utf8]{inputenc}
\usepackage[T1]{fontenc}
\usepackage[english, ngerman]{babel}
\usepackage{libertine}
\usepackage{xstring}
\usepackage{amssymb,amstext,amsmath}
\usepackage{graphicx}
\usepackage{sectsty}
\usepackage{multirow}
\usepackage{dsfont}
\usepackage{amsfonts}
\usepackage{graphics}
\usepackage{float}
\usepackage{dsfont}
\usepackage[hidelinks]{hyperref}
\usepackage{caption}
\usepackage{ifthen}
\usepackage[table]{xcolor}
\usepackage{booktabs}

\hypersetup{
  pdftitle    = {\myTitle},
  pdfauthor   = {\myAuthor},
  pdfcreator  = {pdflatex},
  pdfproducer = {LaTeX with hyperref}
}

\newcommand{\myImage}[2]{
  \begin{center}
    \begin{minipage}{\linewidth}
      \centering
      \makebox[0cm]{\includegraphics[width=\textwidth]{#1}}
    \end{minipage}
    \ifthenelse{ \equal{#2}{}} {
      
    } {
    \captionof{figure}{#2}
    }
  \end{center}
}

\newcommand{\myTitlepage}{%
\begin{titlepage}
  \begin{center}
    \vspace{5cm}
    \huge\bfseries
    \myTitle
    \vspace{1cm}

    \large\normalfont von

    \bigskip
    \textbf{\myAuthor}

    \myDate

    \vspace{2cm}
    \myImage{\myTitleImage}{}
  \end{center}
  \vfill
  \enlargethispage{2cm}
  \parbox[t]{0.45\textwidth}{%
   \myTitleLeft
  }
  \parbox[t]{0.45\textwidth}{\raggedleft%
    \myTitleRight
  }
\end{titlepage}
}

\newcommand{\mySecRef}[1]{%
  \hyperref[sec:#1]{Punkt-}\ref{sec:#1}%
}


\newcommand{\myPackage}[1]{%
  \textit{#1}-Paket%
}

\newcommand{\myFormat}[1]{%
  \textit{#1}%
}

\newcommand{\myPath}[1]{%
  \textit{#1}%
}

\newcommand{\myTitle}{\LaTeX - Grundlagen}
\newcommand{\myAuthor}{Dominik Wille}
\newcommand{\myDate}{22 Oktober 2013}
\newcommand{\myTitleImage}{images/typesetting} %This Image is GNU GPL licenced.
\newcommand{\myTitleLeft}{%
   Freie Universität Berlin\\
   Zentraleinrichtung für Datenverarbeitung\\
   Betriebsysteme und Programmieren%
}
\newcommand{\myTitleRight}{%
  Dozent: \\
  Dr.\,Herbert Voß
}

\begin{document}
\myTitlepage
\section{Vorwort}
Bei diesem Dokument handelt es sich um eine, im Rahmen des ABV-Kurses Betriebsysteme und Programmieren der
Freien-Universität-Berlin, angefärtigte Hausarbeit, die die erlernten Grundlagen von Latex zusammenfasst.

Im folegenden werde ich sowohl auf den Textsatz mit Latex, als auch auf einen meiner Meinung nach sinnvollen
Arbeitsweg mit Latex eingehen.

Desweiteren möchte ich dieses Dokument möglichst so gestalten, dass viele wiederkehrende Elemete in Latex in
Makros gekapselt werden, und somit für weitere Dokumente wiederverwendet werden können.

\section{Dokumentenklassen}
Jedes Latex-Dokument verwendet eine sogenannte Dokumentenklasse. Diese gibt ein Grundgerüst des Dokuments,
wie z.B. Seitenrandabstände, Schriftgrößen und Einrückungsverhalten vor. Definiert wird die Dokumentenklasse
mit dem Kommando \verb+\documentclass{...}+. Die wichtigsten Dokumentenklassen für den deutschsprachigen
Raum sind:
\begin{description}
\item[scrartcl] Klasse für kurze Artikel.
\item[scrreprt] Klasse für lange Berichte.
\item[scrbook] Klasse für Bücher.
\end{description}
Die meisten Dokumentenklassen ermöglichen es außerdem, gewisse Parameter optional zu verändern. Ich verwende in
diesem Dokument bespielsweise
\begin{verbatim}
\documentclass[a4paper, parskip=half]{scrartcl}
\end{verbatim}
um die Seitenränder auf europäische Normen einzustellen und um Absätze als Leerzeile mit halber Höhe zu setzen.

Mit \verb+\maketitle+ bietet Latex auch eine möglichkeit einen Dokumentenkopf zu erzeugen. Die verwendeten
Informationen werden vorher in der Präambel mit \verb+\title{...}+, \verb+\auther{...}+ und
\verb+ \date{...}+ definiert. Möchte man eine eigene Titelseite gestalten, kann man das zu Beginn des Dokuments
in der 
\begin{verbatim}
\begin{titlepage}
  ...
\end{titlepage}
\end{verbatim}
Umgebung machen.

\section{Gliederung}
Latex bietet standardmäßig 7 Ebenen für die Gliederung eines Dokuments. Diese werden durch folgende Kommandos
eingeleitet:
\begin{itemize}
\item \verb+\part{...}+
\item \verb+\chapter{...}+
\item \verb+\section{...}+
\item \verb+\subsection{...}+
\item \verb+\subsubsection{...}+
\item \verb+\paragraph{...}+
\item \verb+\subparagraph{...}+        
\end{itemize}

Um ein Inhaltsverzeichnis zu erzeugen reicht es \verb+\tableofcotents+ aufzurufen. Dieses Kommando erzeugt
automatisch ein Inhaltsverzeichnis der Sektionen des Latex-Dokuments.

\section{Textsatz}
Historisch bedigt ist Tex bzw. Latex vorallem für einen optimierten Textsatz konzipiert. Beim Schreiben eines
Latex-Dokuments ist es bis auf doppelte Leerzeilen, die einen Paragraphen erzeugen nicht relevant, wie die
Quelldatei formatiert ist. Sehr oft ist es hilfreich, eine automatische Worttreenung zu aktivieren, um einen
schönen Blocksatz zu erreichen. Das \myPackage{babel} liefert für viele Sprachen eine automatische
Silbenerkennung, die eine korrekte Worttrennung ermöglicht. Um das Paket einzubinden genügt ein simples
\begin{verbatim}
\usepackage[english, ngerman, ...]{babel} 
\end{verbatim}
in der Präambel. Einer der großen Vorteile bei der Arbeit mit Latex ist, dass man ein semantisch korrektes
Dokument schreibt und die Darstellung des Dokuments ersteinmal nichts mit dessen Inhalt zutun hat. Um Teile
des Textes \textit{kursiv}, \textbf{fett} oder \underline{unterstrichen} darzustellen, bietet sich die Verwendung 
von den einfachen Kommandos 
\begin{verbatim}
\textbf{fett gesetzter Text}
\textit{kursiv gesetzter Text}
\underline{unterstricherner Text}
\end{verbatim}
an. Um Textteile so auszugeben, wie sie im Tex-Dokument stehen, gibt es die \textit{verbatim}-Umgebeung. 
\\ \\
\verb+\begin{verbatim}+\\
\verb+  ...+\\
\verb+\end{verbatim}+\\
\\
Möchte man einzelne Kommandos o.Ä. im Fließtext unterbringen, kann man das mit dem \verb+\verb+ Kommando tun.
Dieses Kommando erwartet einen Delimiter um den wortgetreuen Text zu begrenzen.
\begin{verbatim}
\verb+format C:+
\end{verbatim}
Dieses Beispiel gibt \verb+format C:+ aus. Das \verb#+# ist hierbei der Delimiter.

\section{Bilder}
Bilder können in Latex mit dem \myPackage{graphicx} eingebunden werden, dieses unterstützt gängige
Bildformate wie \myFormat{png, jpeg} aber auch Graphiken im \myFormat{pdf} Format können eingebunden werden.
Das zugehöhrige Kommando lautet:
\begin{verbatim}
\includegraphics{/pfad/zur/bilddatei}
\end{verbatim}
Zu beachten ist dabei, dass die Dateiendung nicht anggeben werden muss. Um also beispielsweise \myPath{bsp.jpeg}
einzubinden reicht \verb+\includegraphics{bsp}+.

Ich benötige für meine Physikprotokolle oft mittig gesetzte Bilder, die ggf. auch über den Text heraustehen.
Ich verwende hierfür folgende Umgebung:
\begin{verbatim}
\begin{center}
  \begin{minipage}{\linewidth}
    \centering
    \makebox[0cm]{\includegraphics[width=...]{/pfad/zur/bilddatei}}
  \end{minipage}
\end{center}
\end{verbatim}

Um eine Bildunterschrift hinzuzufügen gibt es das \myPackage{caption}, eine Unterschrift kann hiermit einfach
durch ein \verb+\captionof{figure}{...}+ hinzugefügt werden.

Da es sich nun um alles andere als ein kurzes, einfaches Kommando handelt, bietet es sich an diesen Block als neues
Kommando zu definieren, siehe hierfür \mySecRef{customCommands}.

\section{Mathematiksatz}
Da ich im Physikstudium diverse Protokolle mit viel mthematischem Inhalt schreibe, möchte ich auch die
Grundlagen vom Mathematiksatz in Latex erklären.

Auch ohne zusätzliches Paket bietet Latex schon einen Mathematikmodus, der durch eine \verb+$...$+
Umgebung eingeleitet und beendet wird. als Beispiel liefert
\begin{verbatim}
$a^2 + b^2 = c^2$
\end{verbatim}
folgende Ausgabe:
$a^2 + b^2 = c^2$. Um auch ganze Rechnungen oder Formelsammlungen in Latex einzubetten, gibt es das
\myPackage{amsmath}. Dieses stellt eine Umgebung für Gleichungen und für Formelblöcke zur Verfügung.
Benutzt werden diese wie folgt:
\begin{verbatim}
\begin{equation}
  a^2 + b^2 = c^2
\end{equation}

\begin{align}
  0       &= x^2 + px + q \\
  0       &= x^2 + px + \frac{1}{4}p^2 - \frac{1}{4}p^2 + q \\
  \frac{1}{4}p^2 - q &= \left(x + \frac{1}{2}p \right)^2 \\
  \pm \sqrt{ \frac{1}{4}p^2 - q} &= x + \frac{1}{2}p \\
  x_{1/2} &= \frac{1}{2}p \pm \sqrt{\left(\frac{1}{2}p\right)^2 - q}  
\end{align}
\end{verbatim}
Was die folgende Ausgabe erzeugt:
\begin{equation}
  a^2 + b^2 = c^2
\end{equation}

\begin{align}
  0                  &= x^2 + px + q \\
  0                  &= x^2 + px + \frac{1}{4}p^2 - \frac{1}{4}p^2 + q \\
  \frac{1}{4}p^2 - q &= \left(x + \frac{1}{2}p \right)^2 \\
  \pm \sqrt{ \frac{1}{4}p^2 - q} &= x + \frac{1}{2}p \\
  x_{1/2}            &= \frac{1}{2}p \pm \sqrt{\left(\frac{1}{2}p\right)^2 - q}  
\end{align}

\section{Listen}
Es gibt in Latex drei Typen von Listen \verb+itemize+, \verb+enumerate+ und \verb+description+, die für
verschiedene Verwendungszwecke konzipiert sind:
\begin{description}
\item[itemize] wird für unsortierte Listen oder Aufzählungen verwendet; die einzelnen Elmente sind alle
  gleich ausgezeichnet.
\item[enumerate] wird für sortierte Listen oder Aufzählungen verwendet; die einzelnen Elemente werden
  nummeriert oder mit aufsteigenden Buchstaben versehen.
\item[description] wird für Listen von Elementen mit Namen verwendet. 
\end{description}
Listen können natürlich verschachtelt werden, wobei sich die Darstellung der Listenpunkte je nach Verschachtelung
ändert. Hier ein Beispiel:
\begin{verbatim}
\begin{enumerate}
\item
  \begin{enumerate}
  \item eins
  \item zwei
  \item drei
  \end{enumerate}
\item
  \begin{itemize}
  \item bla
  \item blu
  \item
    \begin{itemize}
    \item ha
    \item hi
    \end{itemize}
  \end{itemize}
\end{enumerate}
\begin{description}
\item[Delphin] Säugetier, dass im Meer lebt.
\item[Pinguin] Vogel, der am Südpol lebt. 
\end{description}
\end{verbatim}
\newpage
Dieses Beispiel erzeugt die folgende Ausgabe:
\begin{enumerate}
\item
  \begin{enumerate}
  \item eins
  \item zwei
  \item drei
  \end{enumerate}
\item
  \begin{itemize}
  \item bla
  \item blu
  \item
    \begin{itemize}
    \item ha
    \item hi
    \end{itemize}
  \end{itemize}
\end{enumerate}
\begin{description}
\item[Delphin] Säugetier, dass im Meer lebt.
\item[Pinguin] Vogel, der am Südpol lebt. 
\end{description}

\section{Tabellen}
Selbstverständlich bietet Latex auch die Möglichkeit, Tabellen zu erstellen. Die wichtigesten Typen sind:
\begin{description}
\item[tabular] Gewöhnliche Tabellen
\item[longtable] Tabellen, die über mehrere Seiten reichen.
\item[tabularx] Tabellen mit fester Gesamtbreite.
\end{description}
Die Ausrichtung des Inhalts einer Spalte wird in der Tabellen-Umgebeung definiert. Zeilen werden durch ein
\verb+\\+ umgebrochen und Felder mit einem \verb+&+ abgegrenzt. Eine Tabelle in Latex sieht z.B. so aus:
\begin{verbatim}
\begin{center}
  {
    \rowcolors{3}{}{blue!30}
    \begin{tabular}{c c c r}\\\toprule
      Spalte 1 & Spalte 2 & Spalte 3 & \# \\\midrule
      a & b & c & 1\\
      a & b & c & 2\\
      a & b & c & 3\\
      a & b & c & 4\\
      a & b & c & 5\\\bottomrule
    \end{tabular}
  }
\end{center}
\end{verbatim}

\begin{center}
  {
    \rowcolors{3}{}{blue!30}%
    \begin{tabular}{c c c r}\\\toprule
      Spalte 1 & Spalte 2 & Spalte 3 & \# \\\midrule
      a & b & c & 1\\
      a & b & c & 2\\
      a & b & c & 3\\
      a & b & c & 4\\
      a & b & c & 5\\\bottomrule
    \end{tabular}
  }
\end{center}

Bei diesem Beispiel wurde das \myPackage{booktabs} benutzt, das Tabellen optimiert für Bücher bereit stellt. Um
die Hintergrundfarbe alternieren zu lassen gibt es im \myPackage{xcolor}:
\begin{verbatim}
\rowcolors{start}{farbe1}{farbe2}
\end{verbatim}
 Hierfür muss das Paket folgendermaßen geladen werden:
\begin{verbatim}
\usepackage[table]{xcolor}
\end{verbatim}

\section{Fußnoten}
Gerade bei Tabellen, in denen man den Inhalt der Felder möglichst kurz halten will, kann es hilfreich sein
Fußnoten zu verwenden. Aber auch im normalen Fließtext macht es das Schriftbild durch die Vermeidung von Klammern
schöner und besser lesabr. Eine Fußnote kann einfach über das \verb+\footnote{...}+-Kommando hinzugefügt werden
Die Fußnote wird dann auf der aktuellen Seite im unteren Bereich angezeigt.
\begin{verbatim}
\footnote{Dieses Beispiel erzeugt die Fußnote im unetren Seitenrand}
\end{verbatim}
Als Beispiel gebe ich diese Fußnote\footnote{Dieses Beispiel erzeugt die Fußnote im unetren Seitenrand} im
Fließtext.

\section{Eigene Kommandos}\label{sec:customCommands}
Ein neues Kommando kann mit \verb+\newcommand[anzahl parameter]{\kommando}{...}+ definiert werden.
Die übergebenen Parameter können mit \verb+#1+, \verb+#2+, ... verwendet werden. Auch vorhandene Kommandos
können mit \verb+\renewcommand[anzahl parameter]{\kommando}+ umgeschrieben werden. Als Beispiel kann man mit
\begin{verbatim}
\newcommand{\fett}[1]{%
  \textit{#1}%
}
\end{verbatim}
nun mit einem Aufruf von \verb+\fett{...}+ Text fett setzten.

Kommandos können sehr Praktisch sein, um wiederkehrende Code-Blöcke zu kapseln, aber auch um ein Dokument
semantisch zu schreiben. Ich benutze bespielsweise in diesem Dokument für alle Paketnamen, Dateipfade und
andere wiederkehrende Elemente Kommandos, weil ich beim Schreiben des Dokuments noch nicht genau weiß, wie ich diese
setzen möchte, aber auf jeden Fall alle gleich sein sollen. 

Eine andere Anwendung besteht darin, Kommandos dafür zu benutzen, ganze Dokumententeile zu gestalten, und im
eigentlichen Dokument nur die Kommandos aufzurufen. Da das Dokuemnt in einer solchen Anwendung keinen Inhalt
enthält kann man es als Vorlage wieder verwenden. 

\section{Datein einbinden}
Gerade in größeren Dokumenten kann man schnell die Übersicht verlieren oder man möchte Definitionen oder
Teile des Dokuments nicht im Hauptdokument haben. Außerdem kann bei sehr langen Dokumenten das Übersetzten
von Latex in eine pdf viel Zeit beanspruchen und man möchte nicht bei jeder Änderung das gesamte Dokument
übersetzen. 

Latex bietet dafür zwei Kommandos um sepeparte Datein einzubinden \verb+\input{...}+ und \verb+\include{...}+.
Diese haben kleine Unterschiede:
\begin{description}
  \item[input] fügt die Kommandos aus der angegebenen Datei in die aktuelle Datei ein.
  \item[include] fügt den übersetzten Teil der Datei in das erzeugte Dokuemnt ein; erfordert eine neue
    Seite. 
\end{description}

\section{Verweise/Referenzen}
Die einfachste Möglichkeit in Latex Verweise zu implementieren, ist es den Abschnitten, auf die man verweisen
möchte ein Label zu geben. Das kann man einfach mit dem \verb+\label{...}+ Kommando machen. Verweise können auf
verschiedene Objekte zeigen, z.B. Bilder, Tabellen, Sektionen oder Gleichungen.

Den Verweis kann man einfach mit \verb+\ref{label}+ implementieren, angezeigt wird immer die Nummer des Objekts,
auf das verwiesen wird. Besonders praktisch ist hier auch das \myPackage{hyperref}, mit dessen Hilfe man aktive
Links innerhalb des Dokuments erhält. Um unschöne Änderungen des Textsatzes zu verhindern, lädt man das Paket mit
der \textit{hidelinks}-Option:
\begin{verbatim}
\usepackage[hidelinks]{hyperref}
\end{verbatim}

Für Gleichungen im Mathematikmodus, die man natürlich auch mit einem Label versehn kann, gibt es außerdem noch
den \verb+\eqref+, der eine angepasste Ausgabe mit Klammern bereitstellt. 

\section{Arbeiten im Team}
Eine praktische Eigenschaft von Latex Dokumenten ist, dass man nicht das Dokument an sich, sondern seine erzeugenden
\myPath{*.tex}-Datein austauschen muss. Denn da es sich bei diesen, um simple Textdatein handelt, kann man diverse
Versionierungssysteme für Latex-Dokumente verwenden. 

Ich benutze für alle meine Dokumente \textit{git} zur verteilten Versionierung und habe mir generell angewöhnt
auch alleine kein Latex-Dokument mehr zu schreiben, ohne nach wichtigen Änderungen einen Commit zu machen. Das hat
den Vorteil, dass mir jeder Zeit jemand bei meinem Dokument helfen kann und seine Ändereungen automitisch mit
meinen Änderungen zusammengefügt werden. Außerdem kann ich jede Änderung, die irgendwann commitet wurde später noch
nachvollziehen.

Um effektiv mit \textit{git} zu arbeiten, sollte man die \myPath{.gitignore} dahingehend bearbeiten, dass die vom
Latex-Übersetzungsvorgang erzeugten Datein nicht ausgetauscht werden. Hier meine Konfiguration für
\textit{pdflatex}:
\begin{verbatim}
*.toc
*.aux
*.log
/*.pdf
*.dvi
\end{verbatim}

\section{Fazit}
Nach nun ca. einem Jahr in dem ich Latex benutze, kann ich sagen, dass mein Schreiben von Dokumenten sich nachhaltig
verändert hat. Die Herangehensweise in Latex ist um Längen systematischer als bei WYSIWIG basierter Software und ich
erreiche bei längeren Dokumenten im Endeffekt schneller gute Ergebnisse, da ich mich beim Schreiben nur um den Inhalt
und nicht um die Formatierung kümmere. Außerdem bietet Latex von Haus aus Textsatz auf höchstem Niveau, was bei
anderer Software häufig nicht der Fall ist.

Man muss sicherlich etwas Zeit investieren um Latex vernünftig benutzen zu können, allerdings muss man das auch
bei jeder anderen Software.
\end{document}
